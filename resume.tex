\documentclass[10pt]{article}
\usepackage[utf8]{inputenc}
\usepackage{array, xcolor, bibentry}
\usepackage{hyperref}
\usepackage[margin=3cm]{geometry}
\usepackage{longtable}

\title{\bfseries{\Huge David Keller} \\ Ingénieur développeur}
\date{}

\definecolor{lightgray}{gray}{0.8}
\newcolumntype{L}{>{\raggedleft}p{0.14\textwidth}}
\newcolumntype{R}{p{0.8\textwidth}}
\newcommand{\VRule}{{\color{lightgray}\vrule width 0.5pt}}
\newcommand{\HRule}{\begin{center}{\color{lightgray}\rule{0.5\textwidth}{0.5pt}}\end{center}}

\begin{document}

\maketitle

\begin{minipage}[ht]{0.48\textwidth}
06 23 55 65 02 \\
\href{mailto:david.keller@litchis.fr}{david.keller@litchis.fr} \\
75015 Paris France
\end{minipage}

\vspace{20pt}

\section*{Points forts}
Compétence C++\\
Autonomie, Curiosité, Créativité

\section*{Formation}
\begin{longtable}{L!{\VRule}R}
2006-09
& Diplômé de l'école d'\textbf{ingénieur informatique} et management EFREI.

\\
2004-06
& Diplômé de l'IUT de Cachan Génie Électrique \& \textbf{Informatique Industrielle}.
\end{longtable}

\section*{Compétences}
\begin{longtable}{L!{\VRule}R}
Langages
& \textbf{C++14}, C11, Java 6, C\# (.Net 3.5), Python, Ruby, VHDL

\\
Tests
& Google test, Boost unit-test, JUnit

\\
OS
& Linux, FreeBSD, OpenBSD, Windows

\\
SCM
& \textbf{git}, subversion

\\
Build
& CMake, Autotools, bjam, Qmake, Maven, MSBuild

\\Usines 
& Teamcity, Buildbot, Jenkins, CCNet

\\
Doc
& \textbf{Doxygen}, \LaTeX, HTML

\\
Langues
& Anglais technique

\end{longtable}

\clearpage

\section*{Expériences professionnelles}
\begin{longtable}{L!{\VRule}R}
Depuis 2012 & \textbf{Architecte logiciel chez Enyx - Paris}
\\
& Évolution et création des APIs d'accès aux données de marché
  et passage d'ordre via \textbf{FPGA}.
\end{longtable}

\begin{enumerate}
\item Les premières semaines au sein de l'équipe \textit{Market Data}
  furent dédiées à l'ajout de tests
  unitaires et de correctifs mineurs.

\item Puis j'ai basculé sur la création du nouveau
  produit d'accès et passage d'ordre. Ce dernier abstrait
  l'accélération par FPGA de la pile TCP/IP, la publication
  des ordres ainsi que la normalisation des marchés (CME, KRX).
  A cette occasion j'ai pu étudier les couches basses
  liées au FPGA et proposer des améliorations.

\item Suite à ces propositions, j'ai rejoint l'équipe \textit{R\&D}
  où je suis à présent en charge de trouver des nouvelles pistes pour
  faire baisser les latences des produits.
  J'ai étudié notamment les bus (PCIe, QPI, AXI)
  afin de concevoir une couche de communication légère.
  Afin de réaliser cette couche, composée de pilotes Linux et
  bibliothèques utilisateurs, j'ai au préalable conçu un simulateur
  couplant \href{http://qemu.org}{\bf Qemu} à
  \href{http://ghdl.free.fr/}{\bf GHDL} afin d'accélérer le développement
  logiciel \& matériel et permettre la mise en place d'une validation continue.
\end{enumerate}

\HRule

\begin{longtable}{L!{\VRule}R}
2009 - 2012 & \textbf{Prestation à la Société Générale - La défense}
\\
& Évolution et maintenance des composants C++, C\#, java
  de la chaîne de valorisation des deals sur \textbf{grille de calculs}.
\end{longtable}

\begin{enumerate}
\item Dans l'équipe \textit{TECH} de l'\textit{AGREG}.
  Comme son nom l'indique, les tâches étaient résolument techniques et
  légèrement décorrélées de la finance.

\item Ma première mission a porté sur la prise en compte des fuseaux
  horaires par le système.

\item La seconde sur la mise en place d'un cache de données statiques 
  sur le système de fichiers distribué.

\item Les suivantes concernaient avant tout la création de nouvelles
  bibliothèques utilisées par les équipes orientées finance
  afin qu'elles puissent charger plus facilement leurs données
  (accès BDD et caching) et communiquer leurs résultats ou éventuelles erreurs.

\item En marge de ces projets, j'ai oe{}uvré à la mise à jour des processus de
  développement, en particulier la migration \& formation des équipes
  pour le passage à Git.
  Ainsi que la revue de code systématique pour les développeurs débutants
  ou peu rigoureux.
\end{enumerate}

Durant cette période, j'ai expérimenté:
\begin{itemize}
\item Le monde de la finance
\item Utilisation du C++11
\item Le travail en équipe
\item Les problématiques de calcul sur grille
\item Le développement sous Windows
\item Le poids des processus dans un grand groupe
\end{itemize}

\clearpage

\begin{center}
$\hookrightarrow$
\end{center}

\begin{longtable}{L!{\VRule}R}
2009 & \textbf{byVOLTA - Paris}
\\
& \textbf{Conception d'un langage objet}, proche de l'anglais,
  destiné à décrire
  les interactions entre les produits de l'entreprise, ainsi que son
  interpréteur en C++ modulaire, et par ce biais, extensible.
\end{longtable}

\begin{enumerate}
\item Lors du stage M2: seul développeur dans une société de 6
  désigneurs sensoriels, ma responsabilité portait sur
  l'écriture du firmware de \textit{LaChose}, un boitier intelligent
  permettant par exemple aux clients de \textit{byVOLTA} d'orchestrer 
  la communication entre l'éclairage, un diffuseur d'odeur et un 
  équipement audio.

\item Après une semaine de familiarisation avec la gamme de produit,
  j'ai très rapidement été confronté aux questions techniques et reçu
  assez de liberté pour avoir le choix de l'interface d'utilisation 
  du client. J'ai alors opté pour la création d'un langage de script,
  dont l'interpréteur a été implementé en C++ dans les semaines 
  qui ont suivi.
\end{enumerate}

Ce contexte m'a aidé à améliorer:
\begin{itemize}
\item Ma communication avec les non-développeurs.
\end{itemize}

\HRule

\begin{longtable}{L!{\VRule}R}
2006 & \textbf{dBResearch - Liverpool}
\\
& Conception et réalisation de la pile réseau TCP/IP d'une boite noire 
  maritime (ATMEL ARM9) ainsi qu'un serveur http de contenu statique.
\end{longtable}

\begin{enumerate}
\item Dans le cadre du stage L2: seul, j'ai été chargé la première semaine de
  comprendre le fonctionnement d'un cpu arm ainsi qu'un contrôleur ethernet.

\item Puis d'écrire les deux suivantes en C89 un firmware destiné à 
  les piloter.

\item Enfin les deux dernières semaines ont été consacrées à la réalisation
  d'une pile TCP/IP chargée de gérer une connexion ainsi
  qu'un serveur http communiquant des statistiques propres à la pile.
\end{enumerate}

Ce stage m'a permis de renforcer:
\begin{itemize}
\item Mon anglais
\item Mon autonomie
\end{itemize}

\section*{Projets personnels}
\begin{longtable}{L!{\VRule}R}
Github
& \href{https://github.com/DavidKeller/kademlia}{{\bf DavidKeller/Kademlia (C++11)}}
\\
& Table de hashage distribuée simple et couverte par les tests unitaires.

\end{longtable}

\end{document}

