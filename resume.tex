\documentclass[10pt]{article}
\usepackage[utf8]{inputenc}
\usepackage{array, bibentry}
\usepackage[svgnames]{xcolor}
\usepackage[hidelinks]{hyperref}
\usepackage[margin=3cm]{geometry}
\usepackage{longtable}
\usepackage{fontawesome}
\usepackage{tikz}
\usepackage{SIunits}

\definecolor{lightgray}{gray}{0.8}
\newcolumntype{L}{>{\raggedleft}p{0.14\textwidth}}
\newcolumntype{R}{p{0.8\textwidth}}
\newcommand{\VRule}{\color{lightgray}\vrule width 0.5pt}
\newcommand{\HRule}{\begin{center}{\color{lightgray}\rule{0.5\textwidth}{0.5pt}}\end{center}}
\newcommand{\cvtag}[1]{
  \tikz[baseline]
  \node[anchor=base, draw=DarkSlateGray, rounded corners=0.5ex,
        inner xsep=0.5ex, inner ysep=0.5ex,
        text height=2ex, text depth=.25ex]{\color{DarkSlateGray}#1};
}

\begin{document}

\title{\bfseries{\Huge David Keller} \\ Architecte Logiciel - R\&D}
\date{}
\author{}

\maketitle

\begin{center}
\faPhone \hspace{1pt} 06 23 55 65 02
\hspace{5pt} 
\faAt \hspace{1pt} \href{mailto:david.keller@litchis.fr}{david.keller@litchis.fr}
\hspace{5pt} 
\faMapMarker \hspace{1pt} 92100 Boulogne Billancourt - France
\end{center}

\begin{center}
"\textit{Ils ne savaient pas que c’était impossible, alors ils l’ont fait.}"
\end{center}

\section*{Points forts}
Autonomie, Créativité, Curiosité (Passionné)

\section*{Formation}
\begin{longtable}{L!{\VRule}R}
2006-09
& Diplômé de l'école d'\textbf{ingénieur informatique} et management EFREI.

\\
2004-06
& Diplômé de l'IUT de Cachan Génie Électrique \& \textbf{Informatique Industrielle}.
\end{longtable}

\section*{Compétences}
\begin{longtable}{L!{\VRule}R}

Langues
& Anglais technique
\end{longtable}

\clearpage

\section*{Expériences professionnelles}

\subsection*{\textbf{Enyx}}
\textbf{\color{MidnightBlue}Architecte logiciel - R\&D} \\
\faCalendar \hspace{1pt} Depuis 2012 \hspace{1pt} \faMapMarker
\hspace{1pt} Paris  \\

\begin{center}
\cvtag{Kernel Linux} \cvtag{C++11} \cvtag{C} \cvtag{Python} \cvtag{Gitlab}
\cvtag{CMake} \cvtag{Jenkins} \cvtag{Grafana} \cvtag{TDD} \cvtag{BDD}
\cvtag{Réseau} \cvtag{VHDL} \cvtag{Verilog} \cvtag{HLS} \cvtag{QEMU}
\cvtag{Market Data} \cvtag{Passage d'ordre} \cvtag{Doxygen} \cvtag{Sphinx}
\end{center}

\begin{enumerate}
\item Les premières semaines au sein de l'équipe \textit{Market Data}
  furent dédiées à l'ajout de tests unitaires, de correctifs mineurs
  et d'outils de test pour les IPs \textit{VHDL}.

\item Puis j'ai basculé sur la création du nouveau
  produit d'accès et passage d'ordre. Ce dernier abstrait
  l'accélération par \textit{FPGA} de la pile \textit{TCP/IPv4}, la publication
  des ordres ainsi que la normalisation des marchés.
  A cette occasion j'ai pu étudier les couches basses
  liées au FPGA et proposer des améliorations.

\item Suite à ces propositions, j'ai rejoint l'équipe \textit{R\&D}
  où j'étais en charge de trouver des nouvelles pistes pour
  faire baisser les latences des produits.
  J'ai étudié notamment les bus (\textit{PCIe}, \textit{QPI}, \textit{AXI})
  afin de concevoir une couche de communication légère.
  Pour réaliser cette couche, composée de pilotes Linux et
  bibliothèques utilisateurs, j'ai au préalable conçu un simulateur
  couplant \href{http://qemu.org}{\textit QEMU} (pour faire tourner le pilote
  dans un environment maitrisé) à \href{http://ghdl.free.fr/}{\textit GHDL}
  (pour émuler le FPGA) afin d'accélérer le développement et permettre
  la mise en place d'une validation continue.

\item Puis je me suis tourné vers l'outillage des ingénieurs matériels en 
  spécifiant un système de \textit{synthèse} de firmwares sur une
  ferme de serveurs via \textit{Apache MESOS} afin de mutualiser les
  ressources et gagner en temps de génération.
  Toujours dans la même optique, j'ai récrit en python système de build
  matériel afin de simplifier la gestion des différents outils
  fabricants (Xilinx, Intel).

\item Conséquence du départ du CTO, l'équipe \textit{R\&D} a été dissoute
  et j'ai intégré l'équipe \textit{Techno} où est développée entre autre
  la pile \textit{TCP/IPv4} matérielle. J'ai supervisé techniquement 
  le développement d'une bibliothèque qui permet de décharger la gestion
  du \textit{TCPv4} sur \textit{FPGA} d'une application sans changement
  (via interception des fonctions \textit{BSD sockets}).

\item J'ai également refait le système de build, du logiciel cette fois,
  en réalisant un outil en \textit{bash} capitalisant sur \textit{CMake}
  et le packaging \textit{rpm} \& \textit{deb}, ainsi que la mise en place
  de dépôts internes pour de nombreuses distributions.
  
\item L'opportunité s'est présentée de faire de la recherche appliquée
 via un projet en \textit{\href{https://www.xilinx.com/products/design-tools/vivado/integration/esl-design.html}{HLS Xilinx}}: il fallait découvrir ce nouveau langage et
 réaliser avec une application logicielle/matérielle qui devait répondre 
 en quelques \micro\second{} à des évènements boursiers. 

\item Le projet ayant été un relatif succès, j'ai été chargé d'un autre
 projet plus ambitieux, en \textit{\href{https://www.intel.com/content/www/us/en/software/programmable/quartus-prime/hls-compiler.html}{HLS Intel}}, lui aussi à découvrir.
 J'ai travaillé en totale autonomie à cette occasion, en spécifiant avec
 les clients américains et en gérant deux prestataires,
 dont un anglais, afin qu'ils me déchargent de certaines parties du projet
 et qu'ils montent en compétence sur la \textit{HLS}.

\item Le projet terminé, j'ai enchainé sur la spécification et
 l'écriture d'une couche d'abstraction et de simulation du matériel,
 en \textit{C} \& \textit{C++11}, commune à toutes les équipes.
 La revue systématique de code a été mise en place à cette occasion.
\end{enumerate}

\HRule

\subsection*{Prestation à la Société Générale}
\textbf{\color{MidnightBlue}{Développeur}} \\
\faCalendar \hspace{1pt} 2009 - 2012 \hspace{1pt} \faMapMarker \hspace{1pt} La défense \\
\begin{center}
\cvtag{Grille de calculs} \cvtag{C++11}\cvtag{Finance} \cvtag{Base de code de +15 ans}
\cvtag{MSBuild} \cvtag{TeamCity}
\cvtag{Windows} \cvtag{C\#} \cvtag{Java} 
\end{center}

\begin{enumerate}
\item Dans l'équipe \textit{TECH} de l'\textit{AGREG}.
  Comme son nom l'indique, les tâches étaient résolument techniques et
  légèrement décorrélées de la finance.

\item Ma première mission a porté sur la prise en compte des multiples fuseaux
  horaires par le système (la banque ayant des bureaux sur plusieurs continents).

\item La seconde sur la mise en place d'un cache de données statiques 
  sur le système de fichiers distribué \textit{NetApp}. Le challenge était ici
  d'assurer l'unicité des modifications sur les données via un verrou
  fiable (gestion des crashs de processus) sur le système de fichier.

\item Les suivantes concernaient avant tout la création de nouvelles
  bibliothèques utilisées par les équipes orientées finance
  afin qu'elles puissent charger plus facilement leurs données
  (accès BDD et caching) et communiquer leurs résultats ou éventuelles
  erreurs.

\item J'ai également refondu la gestion d'erreur du système écrit en C++,
  en collaboration avec les équipes support afin de faciliter leurs
  diagnostics en véhiculant un maximum d'informations via les
  \textit{std::exception\_ptr}, et du \textit{XSLT} pour afficher une
  page html les mettant en forme.

\item En marge de ces projets, j'ai \oe{}uvré à la mise
  à jour des processus de
  développement, en particulier la migration \& formation des équipes
  pour le passage à Git, la simplification du build et la division par plus
  de 2 de sa durée.
  Ainsi que la revue de code systématique pour les développeurs débutants
  ou peu rigoureux, puis une phase de formation pour leur expliquer 
  diverses notions telles que la complexité algorithmique.
\end{enumerate}

\subsection*{byVOLTA}
\textbf{\color{MidnightBlue}{Développeur}} \\
\faCalendar \hspace{1pt} 2009 \hspace{1pt} \faMapMarker \hspace{1pt} Paris  \\
\begin{center}
\cvtag{Embarqué} \cvtag{Interpréteur} \cvtag{Linux} \cvtag{C++03} \cvtag{bjam}
\end{center}

\begin{enumerate}
\item Lors du stage M2: seul développeur dans une société de 6
  désigneurs sensoriels, ma responsabilité portait sur
  l'écriture du firmware de \textit{LaChose}, un boitier intelligent
  permettant par exemple aux clients de \textit{byVOLTA} d'orchestrer 
  la communication entre l'éclairage, un diffuseur d'odeur et un 
  équipement audio.

\item Après une semaine de familiarisation avec la gamme de produit,
  j'ai très rapidement été confronté aux questions techniques et reçu
  assez de liberté pour avoir le choix de l'interface d'utilisation 
  du client. J'ai alors opté pour la création d'un langage de script,
  dont l'interpréteur a été implementé en C++ dans les semaines 
  qui ont suivi.
\end{enumerate}

\HRule

\subsection*{dBResearch}
\textbf{\color{MidnightBlue}{Développeur}} \\
\faCalendar \hspace{1pt} 2006 \hspace{1pt} \faMapMarker \hspace{1pt} Liverpool \\
\begin{center}
\cvtag{TCP/IPv4} \cvtag{Embarqué} \cvtag{Anglais} \cvtag{Linux}
\end{center}

\begin{enumerate}
\item Dans le cadre du stage L2: seul, j'ai été chargé la première semaine de
  comprendre le fonctionnement d'un cpu \textit{ATMEL ARM9} ainsi
  que son contrôleur \textit{Ethernet} présents dans une boite noire maritime.

\item Puis d'écrire les deux suivantes en C89 un firmware destiné à 
  les piloter.

\item Enfin les deux dernières semaines ont été consacrées à la réalisation
  d'une pile TCP/IP chargée de gérer une unique connexion ainsi
  qu'un serveur HTTP communiquant des statistiques propres à la pile.
\end{enumerate}

\section*{Projets personnels}
\begin{itemize}
\item \faGithub\hspace{1pt} \href{https://github.com/DavidKeller/kademlia}{\bf DavidKeller/Kademlia (C++11)}

Table de hachage distribuée simple et couverte par des tests unitaires.
\end{itemize}

\end{document}

