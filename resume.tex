\documentclass[10pt]{article}
\usepackage[utf8]{inputenc}
\usepackage{array, xcolor, bibentry}
\usepackage{hyperref}
\usepackage[margin=3cm]{geometry}

\title{\bfseries{\Huge David Keller} \\ Développeur}
\date{}

\definecolor{lightgray}{gray}{0.8}
\newcolumntype{L}{>{\raggedleft}p{0.14\textwidth}}
\newcolumntype{R}{p{0.8\textwidth}}
\newcommand\VRule{\color{lightgray}\vrule width 0.5pt}

\begin{document}

\maketitle

\begin{minipage}[ht]{0.48\textwidth}
06 23 55 65 02 \\
\href{mailto:david.keller@litchis.fr}{david.keller@litchis.fr} \\
92100 Boulogne-Billancourt France
\end{minipage}

\vspace{20pt}

\section*{Points forts}
Bon niveau en C++\\
Autonomie, Curiosité, Créativité

\section*{Formation}
\begin{tabular}{L!{\VRule}R}
2006-09
& Diplômé de l'école d'{\bf ingénieur informatique} et management EFREI.

\\
2004-06
& Diplômé de l'IUT de Cachan Génie Electrique \& {\bf Informatique Industrielle}.
\end{tabular}

\section*{Compétence}
\begin{tabular}{L!{\VRule}R}
Langages
& {\bf C++11 (Boost)}, C89, C\# (.Net 4.0), java 6, ruby

\\
OS
& FreeBSD, OpenBSD, Windows, Linux, OSX

\\
SGBDR
& Oracle, Mysql

\\
SCM
& {\bf git}, subversion

\\
Build
& Autotools, CMake, bjam, Qmake, Maven, MSBuild, Usines (Teamcity, CCNet)

\\
Documentation
& {\bf Doxygen}, LaTex, HTML

\\
Langues
& Anglais technique

\end{tabular}

\section*{Expériences professionelles}
\begin{tabular}{L!{\VRule}R}
Depuis 2010
& {\bf Prestation à la Société Générale - La défense}\\
& Evolution et maintenance des composants C++, C\#, java 
  de la chaîne de valorisation des deals.\\
& Mises à jour des outils et process de dévelopement 
  et d'intégration, en particulier migration + formation des équipes pour le passage
  au SCM git.

\\
2009
& {\bf byVOLTA - Paris}\\
& Conception d'un langage objet, proche de l'anglais, destiné à décrire
les interactions entre les produits de l'entreprise, ainsi que son
interpréteur en C++ modulaire, et par ce biais, extensible.

\\
2006
& {\bf dBResearch - Liverpool}\\
& Conception et réalisation de la pile réseau TCP/IP d'une boite noire maritime (ATMEL ARM9) ainsi qu'un serveur http de contenu statique.
\end{tabular}

\end{document}
